\documentclass[a4paper,10pt,spanish]{article}
\usepackage{babel}
\usepackage{listings}

\lstset{language=Prolog,
frame=single,
basicstyle=\footnotesize,
tabsize=3,
showtabs=false,
showspaces=false,
showstringspaces=false}

\begin{document}

\begin{titlepage}

%opening
\title{{\bf Proyecto: Etapa 1}\\ Inteligencia Artificial\vspace{10mm}}
\author{{\bf Profesor:}\\ Simari, Guillermo\\
{\bf Asistente:}\\ Gomez Lucero, Mauro\\
{\bf Comisi\'on:}\\ Marcovecchio, Diego. LU: 83815\\ Touceda, Tom\'as. LU: 84024}
\date{}

\maketitle

\thispagestyle{empty}

\end{titlepage}

\newpage

\tableofcontents

\newpage

\section{Consideraciones previas}

	En cuanto a la modularizaci\'on, $agent\_bugor.pl$ es el archivo principal del agente, en donde se definen los predicados principales (run, start\_ag, etc), y en el directorio $bugorLib$ se encuentran dos archivos: $auxiliares.pl$, en donde se implementan predicados de debugging, y operaciones que no est\'an relacionadas de forma directa ni con el comportamiento ni con la percepci\'on; y en $percept.pl$ se encuentra la implementaci\'on del predicado update\_state, as\'i como todos los predicados que se desprenden de \'el.

	Adem\'as, se aclara que actualmente el movimiento del agente es aleatorio; es decir, no realiza ning\'un tipo de consideraciones para moverse, simplemente se genera un n\'umero al azar y de acuerdo a \'este el agente decide si gira o avanza.

\section{Percepciones}
	\subsection{Turno}
	
	Dado que el estado del agente (stamina, oro, fight\_skill) es una propiedad que var\'ia a medida que los turnos van pasando, se almacenar\'a el n\'umero de turno actual. De esta forma, se podr\'an diferenciar estrategias que el agente persigue al comienzo del juego, de aquellas decisiones que se toman cuando el juego est\'e m\'as avanzado.
	
	\subsection{Mapa}
	
	Para cada celda del campo $Vision$, el agente mantendr\'a un ``mapa mental'' con el contenido de \'este. Para realizar esto, se guardar\'a $mapa(X, Y, Land)$ para todas las celdas por cada nueva percepci\'on. El campo $Land$ corresponde al segundo elemento de la lista $Vision$ que se recibe como parte de la percepci\'on, y representa el tipo de terreno de la celda (que puede ser agua, tierra, monta\~na, o bosque). Esta representaci\'on interna del mapa del mundo ser\'a utilizada repetidas veces en la fase de movimiento de Bugor, pues resultar\'a fundamental a la hora de encontrar caminos para trasladarse de un punto a otro del mapa.
	
	Adicionalmente, se realizan checkeos para asegurar la consistencia en el recuerdo de los tesoros en cada posici\'on; \'esto se detallar\'a en la secci\'on ``Oro''.

	\subsection{Objetos}
	
	Los posibles objetos que se pueden encontrar en el recorrido del mapa son posadas y oro. Cada uno de ellos ser\'a recordado por el agente de un modo particular, y dicho recuerdo se utilizar\'a en las diferentes estrategias de acci\'on del agente.
	
		\subsubsection{Posadas}
		
		Cuando Bugor encuentre una posada, recordar\'a su existencia utilizando un hecho din\'amico $posada([X, Y])$, siendo $(X, Y)$ la posici\'on de la posada en $(Fila,Columna)$. Dado que las posadas se encuentran est\'aticas en el mapa, una vez que una de ellas es avistada, se recordar\'a siempre.
		
		\subsubsection{Oro}
		
		El oro es manejado delicadamente en la ``memoria'' del agente. Cada tesoro es recordado utilizando el predicado din\'amico \emph{oro(Name, Place, Turn)}, donde ``Name'' es el nombre del tesoro, ``Place'' indica d\'onde fue visto, y ``Turn'', el n\'umero de turno en el que se vi\'o por \'ultima vez. Es muy importante destacar que ``Place'' puede ser una lista [X, Y] (indicando las coordenadas en el mapa donde fue visto dicho tesoro), o bien puede ser el nombre del agente al que se haya visto levantando el tesoro. De esta manera, es sencillo inferir r\'apidamente qu\'e tesoros se encuentran tirados en el mapa, qu\'e tesoros puede tener cada uno de los agentes conocidos, y desde hace cu\'antos turnos que no se tiene informaci\'on actualizada de ellos; y por otro lado, utilizar el formato de representaci\'on mostrado tiene la ventaja de que permite realizar actualizaciones sencillas de los recuerdos de los tesoros sin importar si se encontraban tirados o estaban siendo cargados por alg\'un agente.
		
		A continuaci\'on, describiremos brevemente las consideraciones realizadas para el manejo de la ``memoria'' de dichos tesoros.
		
		El primer caso, y el m\'as trivial, es cuando Bugor ve oro en el mapa; este tesoro ``TX'' es recordado como \emph{oro(TX, PosicionActual, TurnoActual)}, y todos los recuerdos anteriores de dicho tesoro son eliminados (recordando que, gracias a la implementaci\'on elegida, es indistinto si el tesoro estaba siendo cargado por un agente, que si estaba en el suelo).
		
		En cada turno, y por cada posici\'on en el rango de vista del agente, se realiza un checkeo para ver si el oro anteriormente recordado en dicha posici\'on contin\'ua estando en ese lugar; de ser as\'i, el recuerdo del oro se mantiene, \emph{pero se actualiza el turno en el que el oro fue visto por \'ultima vez}. Si, por el contrario, el tesoro ya no se encuentra en la posici\'on en la que se lo recordaba, es borrado de la ``memoria''. Recordemos que cada tesoro tiene un nombre propio, por lo que si se recordaba, por ejemplo, un tesoro t1 en la posici\'on [2,2], y en una nueva visita se encuentra un tesoro diferente en la misma posici\'on, el primer tesoro ser\'a igualmente borrado, y se agregar\'a a la ``memoria'' el nuevo tesoro avistado.
		
		Otro caso a considerar ocurre cuando se ve a un agente en el preciso momento en el que est\'a recogiendo un tesoro TX; dicho tesoro tambi\'en es recordado, pero en lugar de grabar una posici\'on como en los casos anteriores, se utiliza \emph{oro(TX, NombreDelAgente, TurnoActual)} (y se realizan otras acciones que se detallar\'an en la secci\'on de ``Interacci\'on con otros agentes''). La implementaci\'on utilizada para representar el oro permite reemplazar el ``recuerdo'' anterior sin importar si el tesoro estaba anteriormente en el piso o estaba siendo cargado por otro agente. Adem\'as, si se ve a un agente cualquiera tirar oro al suelo, inmediatamente se borra cualquier recuerdo que se tuviera de dicho oro, para pasar a recordar que el tesoro se encuentra ahora en el piso (y otras acciones respecto al agente, que tambi\'en se detallar\'an m\'as adelante).
		
		Por \'ultimo, cada vez que se ve a un agente inconsciente, se elimina el recuerdo de todas las piezas de oro que pose\'ia (y si dichas piezas est\'an en el rango visual de Bugor, ser\'an inmediatamente agregadas a la ``memoria'' con su nueva posici\'on).
		
	\subsection{Interacci\'on con otros agentes}
	
	En caso de encontrarse con otro agente, \'este ser\'a recordado en una lista de agentes. Si bien tambi\'en podr\'ia utilizarse un hecho din\'amico para representar la informaci\'on de cada agente, decidimos mantener una lista porque facilita algunas cuestiones de implementaci\'on.
	
	Para cada uno de los agentes de la lista se mantiene cierta informaci\'on que ser\'a utilizada a la hora de definir el comportamiento de Bugor; naturalmente el nombre del agente, la cantidad de veces que fue visto atacando a otros, la cantidad de tesoros que se cree que el agente carga, y si existe cierta posibilidad de que el agente sea ``lento''. El hecho de que el agente haya sido visto atacando a otros implica que es agresivo, por lo que se evitar\'a a aquellos que tengan una cantidad numerosa de ataques realizados. Si bien esto no es una medida directa para medir el fight\_skill del rival (pues para eso ser\'ia necesario considerar los ataques que efectivamente produjeron da\~no), s\'i permite caracterizar el perfil del agente. Un agente considerado ``agresivo'' tiene mayor probabilidad de aumentar su fight\_skill mientras se encuentra fuera del rango de visi\'on de Bugor. La cantidad de oro que (se cree que) el rival est\'a cargando puede inferirse gracias a la representaci\'on del oro mencionada anteriormente, pero se opt\'o, a modo de desici\'on de dise\~no, por contar con el total para no necesitar recalcularlo permanentemente.
	
	El balance de la cantidad de oro cargado por el agente se modifica en todos los mismos casos que se mencionaron anteriormente para el oro: se restar\'a 1 al valor para un agente si se encuentra un tesoro que se sab\'ia que estaba siendo cargado por \'el, o si se ve a otro agente levantar dicho tesoro del suelo; al ver a un agente levantar un tesoro, este valor aumenta; y si el agente se encuentra inconsciente, el valor vuelve a setearse a 0.
	
	Dado el caso de que Bugor se encuentre con m\'as de un agente conocido, se establece una prioridad que ser\'a utilizada a la hora de decidir c\'omo actuar. La prioridad estar\'a dada por la informaci\'on que se infiera de cada agente; en particular, se trata con mayor prioridad a aquellos agentes que se recuerda que tienen oro, ya que atacar a estos agentes puede resultar no solo en aumentar el fight\_skill, sino tambi\'en en aumentar la cantidad de tesoros en caso de dejarlo inconsciente. Los agentes que son perfilados como ``agresivos'', por el contrario, disminuyen el valor de la funci\'on de prioridad.
	
	Se considera, adem\'as, un caso especial para aquellos agentes que fueron vistos inm\'oviles, (previous\_turn\_action = none), a quienes denominamos ``lentos''. El caso de que alg\'un agente piense durante demasiado tiempo y pierda un turno puede significar cierta ventaja a la hora de atacarlo. Es por esto que, si se presenta este caso, el valor de la funci\'on de prioridad autom\'aticamente se duplica, marcando grandes diferencias en agentes que tengan prioridades para ser atacados similares.		
	
	Por \'ultimo destacamos que, si bien no se puede medir de manera directa la stamina de un agente rival, s\'i es posible estimar una cota superior	de \'esta una vez que el juego se encuentra en una instancia avanzada; conociendo la posici\'on actual de un agente dado, y teniendo una buena representaci\'on interna del mundo, se puede estimar una cota superior de la stamina del agente rival calculando la energ\'ia necesaria para desplazarse desde la posada m\'as cercana hasta su posici\'on actual. Dicha estimaci\'on puede ser pr\'actica a la hora de decidir si atacar o no a un agente. Es importante recordar que esta t\'ecnica deber\'ia ser utilizada una vez avanzado el juego, dado que al principio no se cuenta con una buena representaci\'on interna del mundo (lo que puede implicar no conocer una posada cercana), y que los agentes, al comienzo del juego, aparecen en posiciones aleatorias con su stamina lleno.
	
% 	Otra cuesti\'on importante es determinar un cierto posicionamiento de los agentes en el mapa. A cada agente se le asignar\'a una zona del mapa, de esta forma, en caso de tener baja stamina o no tener fight\_skill alto, el agente priorizar\'a zonas donde no se hayan visto agentes agresivos. 
% 	
% 	Para asignar un agente a una zona, como los agentes se mueven a lo largo del mapa, se asumir\'a la pertenencia a una zona circular de 3 de radio. Si un agente est\'a a distancia Eucl\'idea menor a 3 del centro de una zona ya existente, pasar\'a a pertenecer a dicha zona. La existencias de zonas se ver\'a ligada a la asignaci\'on de al menos un agente a ella. Por ello, el mapa de zonas ser\'a din\'amico con respecto al recorrido del agente y su interacci\'on con el entorno.
	
\end{document}
