\documentclass[a4paper,oneside]{report}
\usepackage[spanish]{babel}
\usepackage[latin1]{inputenc}
\usepackage{fullpage}
\usepackage{listings}
\usepackage{gmverb}
\usepackage[colorlinks=true,urlcolor=black,linkcolor=black]{hyperref}%
\usepackage{graphicx}

\setlength{\parskip}{1ex plus 0.5ex minus 0.2ex}

\lstset{language=,keywordstyle=\ttfamily,stringstyle=\ttfamily}
\lstset{breaklines}

\title{Inteligencia Artificial\\Planeamiento: Strips}

\author{Diego Marcovecchio (LU: 83815)\and Tom�s Touceda (LU: 84024)}

\date{24 de Noviembre de 2010}

\begin{document}
\lstset{frame=single}	
\maketitle
		
\tableofcontents

\chapter{Introducci�n}
\section{Descripci�n}

Este proyecto consiste en la implementaci�n de un planificador STRIPS para un Mundo de Bloques similar al utilizado para los trabajos pr�cticos durante el cuatrimestre. El mundo posee un estado inicial fijo en el que todos los bloques se encuentran apoyados sobre la mesa. El objetivo del planificador es: dado un conjunto de metas, encontrar una secuencia de acciones que permita alcanzar todas ellas en un estado final.

\section{Modularizaci�n}
Dada la sencillez del planificador, a diferencia de los proyectos anteriores, toda la implementaci�n del proyecto se encuentra en el archivo \emph{strips.pl}. M�s adelante en el informe se detallar� cada uno de los predicados utilizados.

\chapter{Estado del mundo}

\section{Relaciones entre bloques}
El estado del mundo se describe mediante un conjunto de predicados que denotan relaciones entre bloques. Dichas relaciones son:

\begin{itemize}
	\item \emph{sobre(A, B)}: indica que el bloque \emph{A} est� inmediatamente arriba del bloque \emph{B}.
	\item \emph{libre(A)}: indica que el bloque \emph{A} no tiene ning�n bloque encima.
	\item \emph{enMesa(A)}: es verdadera cuando el bloque \emph{A} est� apoyada encima de la mesa (es decir, no hay ning�n bloque \emph{debajo} de \emph{A}).
\end{itemize}

\section{Acciones}
Las �nicas acciones que pueden realizarse en este mundo son las acciones de apilar, y desapilar bloques. Presentamos, a continuaci�n, el formato de dichas acciones en la notaci�n de STRIPS:

\begin{itemize}
	\item \emph{apilar(A, B)}: coloca al bloque \emph{A} inmediatamente arriba del bloque \emph{B}
	
	Precondiciones
	\begin{lstlisting}
	[libre(A), libre(B), enMesa(A)]
	\end{lstlisting}
	
	Add\_List
	\begin{lstlisting}
	[sobre(A, B)]
	\end{lstlisting}
	
	Del\_List
	\begin{lstlisting}
	[libre(B), enMesa(A)]
	\end{lstlisting}
	
	\item \emph{desapilar(A, B)}: coloca el bloque \emph{A} que est� encima del bloque \emph{B} sobre la mesa.
	Precondiciones
	\begin{lstlisting}
	[sobre(A, B), libre(A)]
	\end{lstlisting}
	
	Add\_list
	\begin{lstlisting}
	[enMesa(A), libre(B)]
	\end{lstlisting}

	Del\_list
	\begin{lstlisting}
	[sobre(A, B)]
	\end{lstlisting}
	
\end{itemize}


\chapter{Estrategia del planeador}
\section{Decisiones de dise�o}
Es importante destacar que el algoritmo trabaja realcanzando las metas que figuran en la delete\_list de cada acci�n, en lugar de ordenar las acciones para proteger las metas. �sta decisi�n fue tomada debido a que no siempre es posible reordenar las metas para evitar los conflictos. A partir de esta decisi�n, tambi�n podemos concluir que el plan encontrado no necesariamente ser� minimal (pues para ello puede ser necesario realizar dicho reordenamiento).

\section{Algoritmo general}
El planeador est� basado en la estrategia presentada en la secci�n "`The STRIPS Planner"' del cap�tulo 8 del libro \emph{Computational Intelligence: a logical approach} (Poole, Mackworth \& Goebel). Se corrigieron algunos errores del pseudoc�digo all� presentado, y se pas� a utilizar un par�metro m�s en el predicado \emph{achieve\_all/4} para representar las metas que ya se cumplieron. De esta manera, el algoritmo general del planeador es:

\begin{itemize}
	\item Obtener la primer meta de la lista de metas pendientes.
	\item Analizar el conjunto de acciones necesarias para alcanzar dicha meta.
	\item Observar cu�les de las metas alcanzadas previamente fueron deshechas por el segundo paso, y volver a agregarlas a la lista de metas pendientes.
	\item Repetir el algoritmo, quitando de la lista de metas pendientes a todas aquellas que hayan sido alcanzadas por la acci�n realizada.
\end{itemize}

\chapter{Implementaci�n}
\section{Descripci�n de los predicados}
En esta secci�n mostraremos todos los predicados utilizados en la implementaci�n, su signatura, y una descripci�n para cada uno de ellos.

PREDICADORRRRRRRRRRRRRRRRRRRRR

\section{C�digo}
Por �ltimo, detallaremos a continuaci�n el c�digo del planeador:
\end{document}